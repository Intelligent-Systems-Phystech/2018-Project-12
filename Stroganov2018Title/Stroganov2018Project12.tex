\documentclass[12pt,twoside]{article}
\usepackage{jmlda}
%\NOREVIEWERNOTES
\begin{document}

\title
    [Обучение машинного перевода без параллельных текстов] % Краткое название; не нужно, если полное название влезает в~колонтитул
    {Обучение машинного перевода без параллельных текстов}
\author
    [Строганов~А.\,А. Бахтеев$^1$~О.\,Ю.  Стрижов$^2$~В.\,В.] % список авторов для колонтитула; не нужен, если основной список влезает в колонтитул
    {Строганов~А.\,А. Бахтеев$^1$~О.\,Ю.  Стрижов$^2$~В.\,В.} % основной список авторов, выводимый в оглавление
    [Строганов~А.\,А. Бахтеев$^1$~О.\,Ю.  Стрижов$^2$~В.\,В.] % список авторов, выводимый в заголовок; не нужен, если он не отличается от основного
\thanks
    {Работа выполнена при финансовой поддержке РФФИ, проект \No\,00-00-00000.
   Научный руководитель:  Стрижов~В.\,В.
   Авторы: А.В. Грабовой, О.Ю. Бахтеев, В.В. Стрижов, Eric Gaussier, координатор Малиновский Г.С.
    Консультант:  Бахтеев~О.\,Ю.}
%\email
%    {author@site.ru}
\organization
     {$^1$Московский физико-технический институт\par
      $^2$Вычислительный центр им. А.~А. Дородницына ФИЦ ИУ РАН}

\abstract
    {Данная работа посвящена исследованию машинного перевода без использования параллельных текстов. Исследование сконцентрировано на использовании нейросети с несколькими моделями Seq2seq для перевода с одного языка на другой и обратно. Особенностью данных моделей является то, что исходный текст кодируется во внутреннее представление модели, а затем декодируется в текст на другом языке. Две модели Seq2seq имеют общее скрытое пространство. Примером, иллюстрирующим работоспособность данного алгортма, будет использован эксперимент по взаимному переводу с двух похожих языков -- русского и украинского.
    
    \bigskip
    \textbf{Ключевые слова}: \emph {машинный перевод, нейросеть, Seq2seq}.}
\maketitle
\section{Введение}
Благодаря недавним достижениям в области глубокого обучения и наличию крупномасштабных параллельных корпусов, машинный перевод достиг впечатляющей производительности на нескольких языковых парах. Тем не менее, эти модели работают очень хорошо, только если они снабжены огромным количеством параллельных данных в порядке миллионов параллельных предложений.
К сожалению, параллельные корпуса стоят дорого, поскольку они требуют специализированного опыта и часто не существуют для языков с низким уровнем ресурсов. 

Есть несколько подходов к построению оптимального метода обучения. Предлагается использовать рекуррентные нейронные сети c короткой и долгой памятью и нейронные сети, в которых реализовано внимание. В других методах используются нейронные сети, которые осуществляют перевод в два этапа. Такой метода называется Seq2Seq.

Данная работа посвящена последнему методу последовательного перевода. Предлагается с помощью первой рекуррентной нейронной сети, основанной на долгой памяти перевести входящую последовательность в вектор, а с помощью второй перевести этот вектор в выходную последовательность на нужном нам языке. Данный метод позволяет
гораздо быстрее обучить нейронную сеть переводу с одного языка на другой, в связи с использованием ей предыдущего опыта и наличию у нее памяти и внимания. Проверка и анализ метода проводятся с помощью алгоритма BLEU(Bilingual evaluation understudy) для проверки качества текста, переведенного с одного языка на другой на паре языков русский-украинский.

\bibliography{biblioteka}
\bibliographystyle{plain}


%\linenumbers

% Решение Программного Комитета:
%\ACCEPTNOTE
%\AMENDNOTE
%\REJECTNOTE
\end{document}
